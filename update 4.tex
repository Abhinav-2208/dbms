\documentclass{article}
\usepackage[utf8]{inputenc}
\usepackage{graphicx}
\graphicspath{{./images}}
\title{ONLINE ART GALLERY \\ Update 4}
\date{9 Feb 2022}
\author{Abhinav Kumar \\ Roll No: 19111001 }
\begin{document}
\maketitle
\begin{center}
   \section*{\textbf{WEB BASED RDBMS}} 
\end{center}
A database can be understood as a collection of related files. How those files are related depends on the model used. Early models included the hierarchical model (where files are related in a parent/child manner, with each child file having at most one parent file), and the network model (where files are related as owners and members, similar to the network model except that each member file can have more than one owner).
\\
\section*{FUTURE OR RDBMS : }
\begin{itemize}
    \item Data are the values stored in the database. On its own, data means very little.
\item  A database is a collection of tables.
\item Each table contains records, which are the horizontal rows in the table. These are also called tuples.
\item Each record contains fields, which are the vertical columns of the table. These are also called attributes.
\item Fields can be of many different types. There are many standard types, and each DBMS (database management system, such as Oracle or SQL) can also have their own specific types, but generally they fall into at least three kinds - character, numeric and date.
\item The domain refers to the possible values each field can contain (it's sometimes called a field specification).
\item An index is a physical mechanism that improves the performance of a database. Indexes are often confused with keys.
\item A view is a virtual table made up of a subset of the actual tables.
\end{itemize}
\end{document}