\documentclass{article}
\usepackage[utf8]{inputenc}
\usepackage{graphicx}
\graphicspath{{./images}}
\title{ONLINE ART GALLERY \\ Update 3}
\date{2 Feb 2022}
\author{Abhinav Kumar \\ Roll No: 19111001 }
\begin{document}
\maketitle
\begin{center}
   \section*{\textbf{SQL DATABASE MANAGEMENT SYSTEM}} 
\end{center}
Database systems have become ubiquitous across the computing landscape. This is partly because of the basic facilities offered by database management systems: physical data independence, ACID transaction properties, a high-level query language, stored procedures, and triggers. These facilities permits sophisticated applications to ‘push’ much of their complexity into the database itself. The proliferation of database systems in the mobile and embedded market segments is due, in addition to the features above, to the support for two-way database replication and synchronization offered by most commercial database management systems. Data synchronization technology makes it possible for remote users to both access and update corporate data at a remote, off-site location. With local (database) storage, this can be accomplished even when disconnected from the corporate network.
\\
\\
\begin{center}
   \section*{\textbf{SQL is a Relational DATABASE MANAGEMENT SYSTEM}} 
\end{center}
A relational database management system (RDBMS) is a program that lets you create, update, and administer a relational database. Most commercial RDBMS's use the Structured Query Language (SQL) to access the database, although SQL was invented after the development of the relational model and is not necessary for its use.The leading RDBMS products Microsoft's SQL Server. Despite repeated challenges by competing technologies, as well as the claim by some experts that no current RDBMS has fully implemented relational principles, the majority of new corporate databases are still being created and managed with an RDBMS.

\end{document}